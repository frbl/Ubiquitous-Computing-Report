\chapter{Introduction} % (fold)
\label{cha:introduction}
Ubiquitous computing means that computers are used pervasively and unobtrusively; everywhere and anytime ~\cite{Weiser:1993:UC:618984.619973}, presumably to help people. On average, humans spend a third of their lives asleep ~\cite{Lauderdale:2006:Am-J-Epidemiol:16740591}. This leads to an interesting question: what can ubiquitous computing do for people while they sleep? 

In order to determine what can be done for sleeping people, a bit more information about the process of sleeping is in order. Sleep consists of several states ~\cite{Silber:2007fk}, each with their own properties. In Chapter \ref{cha:context}, information about these states and their properties, along with some other information, is described.

What ubiquitous computing can do for people while they sleep, depends largely on the sleep state they are in. Detecting the state people are in is, regrettably, not very straightforward; especially without expensive equipment. This problem, along with a possible solution, is described in Chapter \ref{cha:problem}.

They main part of this report is a description of a system that can help in detecting the sleep state that people are in. It is helpful to understand how this system is designed. A description about the design helps with understanding the system; e.g., it is good to know what the requirements are and what kind of components are involved. The design of the system is described in Chapter \ref{cha:design}.

After describing the design of the application, a closer look at its implementation is warranted. It is interesting to see how, for example, the requirements are fulfilled and how are the difficult aspects of the system are handled by the software. Implementation details can be found in Chapter \ref{cha:implementation}.

Now that the details of the system are known, an evaluation of the created system is in order. It is interesting to see what could and could not be done; and, more importantly, what proved to be most challenging. Another topic that is part of the evaluation is what could or should be done in order to use the application to help people with their sleep. This report is concluded with this evaluation, found in Chapter \ref{cha:evaluation}
% chapter introduction (end)