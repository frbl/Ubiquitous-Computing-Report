\chapter{Problem} % (fold)
\label{cha:problem}
In order to take sleep into account in an ubiquitous computing system, it is necessary to detect a persons' sleep state; this should happen in an unobtrusive manner. Sleep state detection by measuring movement using an accelerometer would enable this; but how can this be accomplished. This problem can be divided into four parts, which will be described in this chapter.

\section{Collecting Data} % (fold)
\label{sec:collecting_data}
For the collection of data from a person sleeping, multiple techniques can be used. However, it is often deemed unwanted to be 
- ubiquitous
- non intrusive (privacy)
- a technique which the users do not see as a burden
% section collecting_data (end)

\section{Measuring Movement} % (fold)
\label{sec:measuring_movement}
There exist different methods for measuring which sleep state a person a person is in, for example, with an \emph{EEG} ~\cite{Itil196976}. In this project this detection is done by measuring the amount of movement a person has overnight. In order to measure the movement a sensor is needed. As described this sensor should be non intrusive and be able to capture the movement from a person. In this case the measurements are done using an accelerometer.

- a means to measure the movement
- accelerometer can be used
- in this case in a mobile phone

% section measuring_movement (end)

\section{From Movement to Sleep State} % (fold)
\label{sec:translate_movement_patterns_to_sleep_states}
Capturing movement is not the only problem; when it is possible to capture movement while a user sleeps, this movement is not yet a sleep state. The measurements have to be classified in some way, to determine in which sleep state the user is in. As described by Hoque et \emph{al}. ~\cite{Hoque:2010:MBP:1921081.1921088} it is possible to distinguish the different sleep states based on the movement a person makes while sleeping.


- we only have movement
- movement has to be converted into sleep state
- movement during states differ

% section translate_movement_patterns_to_sleep_states (end)

\section{Communication of Information} % (fold)
\label{sec:communicating_relevant_information}
For doing the translation from movements to sleep state, a classification has to be done on these measurements.

- Sensor data has to be calculated
- Maybe multiple clients, so some sort of central point
% section communicating_relevant_information (end)

% chapter problem (end)